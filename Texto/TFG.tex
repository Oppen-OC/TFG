\documentclass{article}
\usepackage{graphicx}
\graphicspath{{images/}}
\usepackage{changepage}
\usepackage{float}
\usepackage{caption} % en el preámbulo
\usepackage{textcomp}
\usepackage{amssymb}

% Language setting
% Replace `english' with e.g. `spanish' to change the document language
\usepackage[spanish]{babel}
\selectlanguage{spanish}

% Set page size and margins
% Replace `letterpaper' with `a4paper' for UK/EU standard size
\usepackage[letterpaper,top=2cm,bottom=2cm,left=3cm,right=3cm,marginparwidth=1.75cm]{geometry}

% Useful packages
\usepackage{amsmath}
\usepackage{graphicx}
\usepackage[colorlinks=true, allcolors=blue]{hyperref}

% Para posicionar la imagen en la esquina superior izquierda
\usepackage[absolute,overlay]{textpos}

\title{TFG}
\author{Daniel Alpeñes De Lucca}

% Para el espaciado de la tabla de contenidos
\usepackage[utf8]{inputenc}
\usepackage{tocloft}
\setlength{\cftbeforesecskip}{12pt} % Espacio antes de cada sección
\setlength{\cftbeforesubsecskip}{6pt} % Espacio antes de cada subsección

% Bibliografia
\usepackage{csquotes} % Recomendado para citas
\usepackage[backend=biber,style=apa]{biblatex} % Usa el estilo APA (puedes cambiarlo)
\addbibresource{biblio.bib}

\begin{document}

% Colocar la imagen en la esquina superior izquierda
\begin{textblock*}{5cm}(0.8cm, 1cm) % Tamaño y posición de la imagen
    \includegraphics[width=4cm]{etsinf_logo} % Ajusta el tamaño de la imagen
\end{textblock*}
\begin{textblock*}{5cm}(17cm, 1cm) % Tamaño y posición de la imagen
    \includegraphics[width=4cm]{UPV_logo} % Ajusta el tamaño de la imagen
\end{textblock*}

\begin{titlepage}
\vspace*{2cm} % Ajusta este valor para mover el contenido hacia abajo
\centering
{\bfseries\LARGE Universitat Politècnica de València \par}
\vspace{1cm}
{\scshape\Large Escuela técnica superior de ingeniería informática \par}
\vspace{3cm}
{\scshape\Huge T\'itulo del proyecto \par}
\vspace{3cm}
{\itshape\Large Trabajo de fin de grado \par}
\vfill
\vspace{3cm}
{\Large Autor:\par}
{\Large  Daniel Alpeñes De Lucca \par}
\vspace{1cm}
{\Large Tutoras:\par}
{\Large  Stella María Heras Barberá \par}
{\Large  María Laura Montalban Domingo \par}

\vfill
{\Large Curso académico 2024-2025 \par}
\end{titlepage}

\begin{titlepage}
{\bfseries\LARGE Dedicatoria \par}
Don pollo
\end{titlepage}

{\bfseries\LARGE Agradecimientos \par}
\begin{itemize}
  \item A toda la gente bonita de internet
  \item Xenu
\end{itemize}
\newpage

{\bfseries\LARGE Resumen \par}
Lorem ipsum dolor sit amet consectetur adipiscing elit, duis nostra sagittis nunc class mauris fermentum, semper lobortis eu dui per ridiculus. Sodales augue ad neque lobortis taciti facilisi, nec cum vehicula scelerisque senectus ante, inceptos massa maecenas vel natoque. Faucibus sem mattis sociosqu tempor proin sapien egestas tempus, purus condimentum ligula tellus libero penatibus mauris tortor, sagittis cum aenean nunc rutrum odio habitasse.

Eros sociis dictumst auctor habitasse libero molestie nascetur laoreet sodales, a vitae cubilia sollicitudin hendrerit elementum neque ullamcorper, mollis ultrices felis enim conubia lacus scelerisque mi. Semper orci nisl aliquam imperdiet viverra ac, molestie litora penatibus aliquet himenaeos feugiat conubia, habitasse eu leo volutpat curae. Quam parturient purus accumsan eu dui curae torquent porta ligula, nibh ornare augue aenean mus sem iaculis arcu, et sapien eros volutpat enim feugiat ac metus.

\vspace{1cm}

{\bfseries Palabras clave \par}
\vspace{0.25cm}
LLM • RAG • NLP • SQL • Base vectorial • Query • Prompt • Scrapper • GPQA • Búsqueda semántica
\newpage

\addtocontents{toc}{\protect\enlargethispage{\baselineskip}}
{\Large % Cambia el tamaño de la letra aquí
\tableofcontents
}\newpage

% Secciones y subsecciones

\section{Introducción}
Las agencias públicas encargadas del desarrollo de infraestructuras están demandando modelos digitales de gestión que optimicen la contratación de los contratos de obras. El objetivo de estos modelos es reducir los sobrecostes y retrasos en la construcción, garantizando una mayor eficiencia en la planificación y ejecución de los proyectos.


\subsection{Motivación}
Uno de los principales desafíos en la gestión de contratos de obra pública es la subjetividad en la identificación y evaluación de riesgos. Además, la interdependencia de los riesgos y la incertidumbre inherente a los proyectos de infraestructura complican la toma de decisiones. Por ello, es necesario desarrollar metodologías que permitan minimizar estas incertidumbres y mejorar la precisión en la estimación de los factores de contratación.

\subsection{Objetivos}
El proyecto busca aplicar técnicas de Inteligencia Artificial (IA), como los modelos de lenguaje de gran tamaño (LLMs), el procesamiento del lenguaje natural (NLP) y la argumentación computacional, para abordar los siguientes objetivos:
\begin{itemize}
  \item Desarrollar un algoritmo para extraer información de la licitación y adjudicación de contratos de obras.
  \item Desarrollar un modelo de análisis de la relación calidad-precio de las ofertas presentadas, considerando las características de la contratación
  \item Desarrollar un modelo predictivo de riesgos de sobrecostes o retrasos temporales.
\end{itemize}
Para alcanzar estos objetivos, se utilizarán bases de datos con información sobre expedientes de licitación y ejecución de contratos de la administración pública.

\subsection{Metodología}
La forma de trabajo con la que se llevó a cabo este proyecto fue por medio de reuniones semanales entre mi persona y ambas tutoras, en las que se discutía sobre el progreso realizado y se planteaban nuevos objetivos para la próxima reunión. Este proceso que también apoyado por comunicación constante por medio de mensajería instantánea, agilizando así nuestra comunicación. 
\newpage

\section{Estado del arte}
En los últimos años, el mundo ha experimentado una transformación acelerada debido a la democratización de diversos modelos generativos de inteligencia artificial. Un hito clave en este proceso fue el lanzamiento de ChatGPT el 30 de noviembre de 2022 por la empresa estadounidense OpenAI, lo que marcó el inicio de una de las mayores revoluciones tecnológicas del siglo. Como prueba de su impacto, esta plataforma alcanzó 100 millones de usuarios activos en menos de dos meses, un crecimiento significativamente más rápido que el de aplicaciones como Instagram, que tardó más de dos años en lograr la misma cifra. Desde entonces, la popularidad de los modelos de lenguaje ha seguido en ascenso, generando un cambio radical en múltiples industrias. Un ejemplo de ello se evidenció en mayo de 2024, cuando la empresa china DeepSeek lanzó su propio LLM, lo que ocasionó pérdidas multimillonarias en la bolsa de valores estadounidense. \\
En la actualidad (año 2025), los LLM han alcanzado niveles de sofisticación sin precedentes, al punto de superar pruebas como el test de Turing. Existen cientos de modelos distintos y una amplia gama de tecnologías que aprovechan su capacidad de razonamiento y procesamiento del lenguaje natural, incluyendo traductores en tiempo real, imitadores de voz, generadores de código y asistentes virtuales, entre otros. Sin embargo, ante la gran cantidad de información y herramientas disponibles, puede resultar abrumador identificar la solución adecuada para un problema específico; no obstante, a continuación una serie de tecnologías y herramientas (no todas relacionadas con LLM's) que han demostrado ser útiles para este caso concreto:

\subsection{LLMs}
\textbf{Para empezar, ¿exactamente qué es un LLM?} \\
Un LLM (por sus siglas en inglés, Large Language Model) es un modelo de aprendizaje profundo basado en redes neuronales que ha sido entrenado con enormes cantidades de texto. Gracias a este entrenamiento, es capaz de identificar patrones, relaciones entre palabras y generar respuestas basadas en probabilidades. Sin embargo, es importante destacar que un LLM no <<entiende>> el contenido que genera en el sentido humano, sino que produce respuestas basadas en la información con la que ha sido entrenado; en base a esta información, crea matrices con miles de millones de parámetros que contienen la probabilidad de que una palabra esté seguida por otra. \\

\begin{center}
\includegraphics[width=12cm]{llm_probability}
\end{center}

\setlength{\parindent}{10pt} % Sangria
\textbf{¿Cuales son los mejores LLM?}  \\


\begin{figure}[H] % o [htbp] según prefieras
\centering
\includegraphics[width=15cm]{LLM_Graph.png}
\caption{Gráfico que compara el desempeño de diversos LLMs a fecha de hoy} 
\cite{llm-stats}
\label{tab:LLM_Graph}
\end{figure}

Como se puede ver en la figura 1, al momento de redactar este documento, los LLMs que lideran el mercado son: \textbf{ChatGPT-o3}, \textbf{Claude 3.7 Sonnet} y \textbf{Grok-3}; cada uno con un puntaje GPQA del \textbf{87.7\%}, \textbf{84.8\%} y \textbf{84.6\%} respectivamente. Este porcentaje corresponde al resultado de una serie de pruebas realizadas por Google para evaluar la fiabilidad de cada modelo en la resolución de problemas y consultas en áreas como física, química, programación, entre otras.
Sin embargo -y aunque pueda parecer contraproducente- el LLM elegido para sobrellevar el procesamiento computacional de este proyecto fue \textbf{Llama 3.3 70B Instruct}, por razones que se discutirán en la sección \ref{LLM}.

\newpage

\subsection{Herramientas similares} 
  \subsubsection{DoubleTrade}
  DoubleTrade es una plataforma especializada en la monitorización y gestión de oportunidades de negocio en los sectores de la licitación pública y la construcción, tanto en España como en Francia. Esta compañía ofrece soluciones para empresas que buscan optimizar su participación en concursos públicos, privados y proyectos de obra.

\begin{enumerate}
    \item \textbf{Monitorización de licitaciones y adjudicaciones}
    \newline
        Permite acceso a información detallada y actualizada sobre licitaciones, adjudicaciones, pliegos, acuerdos marco y vencimientos, tanto en España como en Francia, con alertas personalizadas por sector, importe, ubicación, CPV, palabra clave, entre otros.
    
    \item \textbf{Plataforma de gestión de licitaciones}
        \newline
        Cuenta con herramientras capaces de organizar y asignar licitaciones a equipos internos, archivar y dar seguimiento a oportunidades en un entorno colaborativo, e integrar con CRM o recibir datos vía SFTP (protocolo de transferéncia de archivos).

    \item \textbf{Análisis con inteligencia artificial}
    \newline
        Tiene la capacidad de realizar el procesamiento y análisis de pliegos en segundos, destacando criterios clave, además de enviar alertas de vencimientos y nuevas convocatorias.

    \item \textbf{Herramientas de Business Intelligence (BI)}
    \newline
        Realiza estudios estratégicos sobre mercado, tendencias y competencia.
        además de opciones de visualización de datos como: volumen de negocio, materiales prescritos, distribución geográfica.

    \item \textbf{Bases de datos especializadas}
    \newline
        Cuenta con información sobre empresas del sector de construcción y licitaciones públicas, datos financieros, legales y de contacto de adjudicatarios.

    \item \textbf{Cobertura sectorial amplia}
    \newline
        Cubre los siguientes sectores: construcción, sanidad, TIC, facility management, entre otros. Además, cuenta con información tanto de proyectos públicos como privados.

\end{enumerate}

  \subsubsection{Licitaciones.es:} 
  Licitaciones.es es una plataforma especializada en la identificación y gestión de oportunidades de negocio en el ámbito de la contratación pública. Ofrece servicios que permiten a las empresas acceder a información detallada sobre licitaciones públicas, facilitando su participación en concursos y mejorando su competitividad en el mercado.

  \begin{enumerate}
    \item \textbf{Servicio de Alertas de Licitaciones}
    \newline
        Recibe información sobre nuevas oportunidades de negocio, cuenta con la opcion de personalizar perfiles de búsqueda según sector, ubicación y otros criterios, puede visualizar y organizar licitaciones relevantes en un panel de control y recibir notificaciones directas por correo electrónico o teléfono móvil.

    \item \textbf{Informes Personalizados}
    \newline
        Puede elaborar informes de licitación personalizados por expertos, proporcionar informes continuos como herramienta de control y seguimiento, y ofrecer acceso a datos históricos para análisis detallado.

    \item \textbf{Análisis de Competencia y Mercado}
    \newline
        Tiene la capacidad de estudiar patrones de compra de entidades adjudicadoras, analizar la competencia y sus ofertas presentadas, e identificar clientes potenciales y tendencias de mercado.
        
    \item \textbf{Licitaciones Futuras}
    \newline
        Cuenta con detección anticipada de oportunidades de negocio mediante análisis de datos históricos, preparando a la empresa con antelación para futuras licitaciones.

    \item \textbf{Servicios de Consultoría y Asesoramiento}
    \newline
        Ofrece asesoramiento personalizado para optimizar la participación en licitaciones, brinda soporte en la elaboración y presentación de ofertas competitivas, y desarrolla estrategias para mejorar la tasa de éxito en concursos públicos.
        
    \end{enumerate}
    \cite{licitaciones}
  
  \subsubsection{Telecit@:} 
  Telicit@ es una solución tecnológica desarrollada por Tesera de Hospitalidad que facilita a las organizaciones la gestión integral de licitaciones públicas. Permite mantener un repositorio único de información accesible para toda la organización, optimizando el proceso de contratación pública. 
  Sus características principales son: 
\begin{enumerate}
    \item \textbf{Parametrización dinámica del contenido de la documentación del expediente}  
    \newline
    Ofrece un sistema flexible para predefinir la configuración del sistema documental y permite adaptar la documentación a las necesidades específicas de cada expediente.
    
    \item \textbf{Sistema de mensajería para envío de notificaciones}  
    \newline
    Gestiona envíos de información internos y envía notificaciones automáticas sobre cambios en las licitaciones.
    
    \item \textbf{Gestión y seguimiento de expedientes, ofertas, contratos y acuerdos marco}  
    \newline
    Realiza un seguimiento detallado de ofertas, contratos y acuerdos marco, y gestiona cambios de precios, discontinuidad de productos, prórrogas y vencimientos.
    
    \item \textbf{Herramientas para comparar resultados con la competencia}  
    \newline
    Captura resultados de adjudicaciones, precios y puntuaciones de criterios, y apoya la toma de decisiones estratégicas mediante análisis comparativos.
    
    \item \textbf{Accesibilidad y disponibilidad}  
    \newline
    Permite el acceso desde cualquier lugar y en cualquier momento, sin requerir instalaciones locales, sólo un ordenador con acceso a Internet.
    
    \item \textbf{Soporte y seguridad}  
    \newline
    Ofrece respaldo de Tesera de Hospitalidad, empresa líder en gestión de procesos de contratación pública, garantiza la seguridad y confidencialidad de los datos con copias de seguridad periódicas, y proporciona servicio de soporte telefónico y actualizaciones ante cambios legislativos.
\end{enumerate}

  \subsubsection{Tendios:} 
  Tendios es una plataforma digital especializada en contratación pública que facilita el proceso de licitación a empresas, entidades públicas y profesionales. Utiliza inteligencia artificial para ayudar a encontrar, analizar, gestionar y ganar licitaciones públicas, ofreciendo una interfaz intuitiva y funcionalidades avanzadas.
  Sus principales caracteristicas son:

\begin{enumerate}
\item \textbf{Búsqueda y filtrado de licitaciones}  
\newline
Permite el acceso a licitaciones y adjudicaciones de múltiples fuentes oficiales en tiempo real, esto con la ayuda de filtros avanzados por palabras clave, códigos CPV, ubicación y otros criterios, además de búsqueda semántica e inteligencia artificial.

\item \textbf{Alertas personalizadas}  
\newline
Ofrece notificaciones configurables según las necesidades del usuario y permite la recepción de alertas por correo electrónico sobre licitaciones relevantes.

\item \textbf{Análisis de mercado y analítica predictiva}  
\newline
Proporciona dashboards con datos de expedientes de contratación por sector, empresa, cliente o región, y predice ofertantes y precios en nuevas licitaciones.

\item \textbf{Gestión de tareas y oportunidades}  
\newline
Facilita la coordinación de equipos en la preparación de ofertas y permite el control de fechas y el seguimiento de licitaciones asignadas.

\item \textbf{Generación de documentos}  
\newline
Permite la creación de borradores de pliegos técnicos y administrativos a partir de plantillas, y el entrenamiento sobre parámetros de contratación pública e históricos de adjudicaciones.

\item \textbf{Integración con otros sistemas}  
\newline
Ofrece una API para la transmisión de datos a CRMs como Salesforce o Hubspot, e integración con herramientas de análisis como PowerBI.

\item \textbf{Interfaz de usuario intuitiva}  
\newline
Presenta un diseño sencillo y fácil de usar, accesible para usuarios con distintos niveles técnicos, con navegación clara y funcionalidades bien organizadas.

\item \textbf{Soporte y recursos educativos}  
\newline
Proporciona un servicio al cliente dedicado para resolver problemas y responder consultas, además de recursos educativos como blogs y guías para mejorar el conocimiento sobre procesos de licitación.
\end{enumerate}

  \subsubsection{Oclem:} 
  Oclem es una empresa consultora en la gestión de licitaciones públicas en España. Ofrece una plataforma avanzada que centraliza y filtra licitaciones según las preferencias de cada empresa, además de proporcionar servicios de consultoría integral para optimizar la participación en concursos públicos. Sus principales características son: 

\begin{enumerate}
    \item \textbf{Plataforma de licitaciones}
    \newline
    Centraliza todas las ofertas publicadas por las Administraciones Públicas, permite el filtrado personalizado por ubicación, sector o tipo de servicio, realiza un seguimiento exhaustivo de licitaciones en las que se participa, proporciona actualizaciones en tiempo real sobre nuevas licitaciones y cambios en convocatorias, y cuenta con una interfaz intuitiva para facilitar la navegación y gestión.

    \item \textbf{Consultoría especializada}  
    \newline
    Ofrece asesoramiento personalizado por consultores especializados, analiza la viabilidad de expedientes según características empresariales, prepara documentación técnica y económica para ofertas, y gestiona la presentación y seguimiento de ofertas.
    
    \item \textbf{Análisis competitivo}  
    \newline
    Evalúa el posicionamiento frente a otros licitadores e identifica puntos de mejora para futuras convocatorias.
    
    \item \textbf{Clasificación empresarial}  
    \newline
    Asiste en la obtención y renovación de la clasificación necesaria para participar en contratos públicos.
    
    \item \textbf{Soporte y formación}  
    \newline
    Proporciona consultas en línea con diferentes departamentos y ofrece formación para mejorar las habilidades en la preparación de licitaciones.
\end{enumerate}

\subsubsection{Gobierto:} 
Gobierto es una plataforma desarrollada por la empresa Populate que se centra en mejorar la transparencia, la participación ciudadana y el análisis de datos en el ámbito del sector público. Su objetivo principal es facilitar el acceso y comprensión de la información pública tanto para administraciones como para ciudadanos. Sus principales características son:

\begin{enumerate}

    \item \textbf{Busca licitaciones de forma natural}
    \newline
    Utiliza palabras clave y/o CPVs, excluye determinados términos, y combina criterios de búsqueda para obtener resultados más precisos. Nuestro buscador es muy rápido y te permite realizar búsquedas complejas de forma sencilla y rápida.
    
    \item \textbf{Configura alertas}
    \newline
    Crea alertas personalizadas para recibir un resumen de las nuevas licitaciones y las adjudicaciones. Cada persona de tu equipo podrá suscribirse a las alertas que necesite, y en cualquier momento podrás consultar búsquedas guardadas de tus colegas.
    
    \item \textbf{Refina tu búsqueda con facilidad}
    \newline
    Ajusta y refina tu búsqueda con facilidad y agilidad. De esta forma podrás iterar en la definición de tu búsqueda para contemplar todas las palabras clave que necesitas, allá donde los CPVs no son suficientes para monitorizar todas las licitaciones que necesitas.
    
    \item \textbf{Analiza a tu competencia}
    \newline
    Extrae de forma automática una lista de las empresas con las que compites en base a las licitaciones que han ganado. Analiza los precios que han ofertado y entiende cómo se están comportando en cuanto a las bajas históricas y recientes. Identifica sus principales clientes para desarrollar tu estrategia comercial.
    
    \item \textbf{Genera informes}
    \newline
    Nuestra herramienta te permite generar informes detallados basados en tus búsquedas y análisis. Estos informes te proporcionarán una visión clara y comprensible de los datos, lo que te permitirá tomar decisiones informadas y estratégicas.
    
    \item \textbf{Consulta el detalle de licitaciones y adjudicaciones}
    \newline
    Accede a información detallada de cada licitación y adjudicación. Esto incluye el título de la licitación, el CPV, el tipo de contrato, la empresa ganadora, el monto ofertado, entre otros datos. Esta información detallada te ayudará a entender mejor el mercado y a tomar decisiones más informadas.

    \item \textbf{Portal unificado}
    \newline
    Ofrece un entorno web centralizado para acceder a toda la información pública, mejorando la organización y accesibilidad de contenidos como contratos, presupuestos, normativas y resultados de gestión.
\end{enumerate}

  
\subsection{Crítica al estado del arte}
1.- Presencia internacional, 2.- Servicio gratuito, 3.- Asesoramiento profesional, 4.- Notificaciones, 5.- Exportación de datos, 6.- Código abierto, 7.- API, 8.- Búsqueda semántica, 9.- Herramientas IA
\begin{table}[H]
\centering
\resizebox{\linewidth}{!}{%
\begin{tabular}{|l|c|c|c|c|c|c|c|}
\hline
\textbf{ } & \textbf{DoubleTrade} & \textbf{Licitaciones} & \textbf{Telecit@} & \textbf{Tendios} & \textbf{Oclem} & \textbf{Gobierto} & \textbf{Mi TFG} \\
\hline 
\multicolumn{8}{|c|}{\textbf{Herramientas útiles}} \\

\hline
1 & Si & Si/No & Si & Si/No & Si & No & No\\
2 & No & No & No & No & No & No & Si\\
3 & No & Si & No & No & Si & No & No\\
4 & Si & Si & Si & Si & No & Si & ""\\

\hline
\multicolumn{8}{|c|}{\textbf{Indicadores técnicos y de innovación}} \\
\hline
5 & No & No & No & Si & No & Si & Si\\
6 & No & No & No & No & No & No & Si\\
7 & No & No & No & Si & No & Si & ""\\
8 & Si & Si & Si & Si & No & Si & Si \\
9 & Si & No & No & Si & No & Si & Si\\

\hline

\end{tabular}
}
\captionsetup{justification=centering}
\caption{Comparativa útil y técnica entre herramientas similares para licitaciones públicas}

\label{tab:Copmarativa_2}
\end{table}

\subsubsection{Soluciones con IA}
\begin{enumerate}
    \item \textbf{Doubletrade:} DoubleTrade utiliza inteligencia artificial para optimizar la participación en licitaciones públicas, ofreciendo recomendaciones personalizadas basadas en el perfil de cada empresa, análisis detallado de pliegos y documentos, asistencia en la redacción de propuestas, estrategias de precios competitivos mediante modelos predictivos, y gestión automatizada de documentación para garantizar el cumplimiento legal y técnico.
    \newline
    \item \textbf{Tendios:}
    Tendios cuenta con una plataforma basada en inteligencia artificial, la cual optimiza la gestión de licitaciones públicas para empresas y entidades. Integra funcionalidades como interacción en lenguaje natural para consultar pliegos, búsqueda y filtrado inteligente de licitaciones relevantes, análisis predictivo con dashboards de mercado, generación automática de documentos, y conexión con sistemas externos como CRMs y herramientas de análisis.
    \newline
    \item \textbf{Gobierto:} 
    Gobierto utiliza inteligencia artificial para optimizar la contratación pública mediante modelos predictivos que estiman el número de ofertas para licitaciones, extracción y clasificación automática de texto en pliegos, y herramientas como Gobierto Redactor, un editor colaborativo asistido por IA para redactar cláusulas. Además, ofrece Gobierto Asistente, un chatbot especializado en contratación pública que responde preguntas basándose en fuentes oficiales, junto con funciones de NLP como recomendación de CPVs, resúmenes automáticos y sistemas de preguntas-respuestas sobre normativa.
    \item \textbf{Mi TFG:} 
    La herramienta emplea la técnica RAG para extraer datos importantes del documento y mostrarlos de manera organizada en tablas, además de mostrar las fuentes de estos datos para garantizar su fiabilidad. 

\end{enumerate}


Si bien queda claro que las opciones son amplias y variadas, siguen existiendo carencias, las cuales este proyecto pretende abordar:

\begin{itemize}

\item{Transparencia de los datos:}
Otras herramientas o no validan su información o simplemente citan el documento completo de donde han extraido la información, este programa cuenta con la capacidad de citar directamente las fuentes de donde ha obtenido la información, especificando la página o páginas que ha utilizado para formular su respuesta.

\item{Código abierto:}
Todos los programas y servicios listados privatizan por completo el uso de sus herramientas, mientras que la funcionalidad de mi herramienta es completamente transparente y maleable para cualquier programador que quiera hacer uso de ella.

\item{Precio:}
La mayoría de herramientas listadas no permiten un acceso total a sus funcionalidades a menos de que se pague una cuota (que en algunos casos puede llegar a los miles de euros), en cambio, esta herramienta es completamente gratiuta y de código abierto, democratizando asi el acceso a la información.

\end{itemize}

\subsection{Propuesta}
Lorem ipsum dolor sit amet consectetur adipiscing elit, duis nostra sagittis nunc class mauris fermentum, semper lobortis eu dui per ridiculus. Sodales augue ad neque lobortis taciti facilisi, nec cum vehicula scelerisque senectus ante, inceptos massa maecenas vel natoque. Faucibus sem mattis sociosqu tempor proin sapien egestas tempus, purus condimentum ligula tellus libero penatibus mauris tortor, sagittis cum aenean nunc rutrum odio habitasse.

Eros sociis dictumst auctor habitasse libero molestie nascetur laoreet sodales, a vitae cubilia sollicitudin hendrerit elementum neque ullamcorper, mollis ultrices felis enim conubia lacus scelerisque mi. Semper orci nisl aliquam imperdiet viverra ac, molestie litora penatibus aliquet himenaeos feugiat conubia, habitasse eu leo volutpat curae. Quam parturient purus accumsan eu dui curae torquent porta ligula, nibh ornare augue aenean mus sem iaculis arcu, et sapien eros volutpat enim feugiat ac metus.

\newpage


\section{Análisis del problema}
Una vez concluido el estudio del estado del arte, resulta fundamental realizar un análisis detallado del problema específico que aborda este proyecto. El objetivo de esta sección es ofrecer una visión clara y profunda de los retos asociados a la extracción, procesamiento y presentación de información contenida en documentos PDF y HTML, utilizando técnicas avanzadas de procesamiento de lenguaje natural y modelos de lenguaje (LLM).

En este contexto, se identifican los principales desafíos relacionados con la heterogeneidad de los formatos documentales, la complejidad estructural de los archivos PDF y HTML, y la necesidad de transformar dicha información en datos estructurados y útiles para el usuario final. Se establecerán los requisitos funcionales y no funcionales que debe cumplir la solución, así como los criterios de calidad y eficiencia que guiarán su desarrollo.

Además, se modelará conceptualmente la arquitectura del sistema, detallando los procesos de extracción, segmentación y almacenamiento de la información, así como los mecanismos para su posterior consulta y visualización. Este análisis permitirá anticipar posibles dificultades técnicas y sentar las bases para la implementación de una solución robusta, escalable y capaz de incorporar futuras mejoras e innovaciones en el ámbito del procesamiento automático de documentos.

\subsection{Identificación y análisis de soluciones posibles}
A groso modo, la herramienta pretende identificar ciertos expedientes de la página de contrataciondelestado.es y extraer información de los mismos, abordando este problema se han manifestado las siguientes problemáticas: 
\subsubsection{Problemáticas:}
\begin{enumerate}
    \item \textbf{Volumen y almacenamiento de datos:}
    Actualmente solo en la comunidad valenciana existen cerca de 13.000 expedientes listados en la plataforma de contrataciondelestado.es.    
    \item \textbf{Varianza interna:}
    Dentro de la propia plataforma (donde existe un formato relativamente estandarizado) hay muchas variables a tomar en cuenta, por ejemplo, dentro de la página de cada expediente el número de secciones que contienen información varía, normalmente suele aparecer primero un resumen con los campos más generales del contrato, luego  un área relacionada con el proceso de contratación en si y por último un área donde se listan documentos relacionados.
    \newline Esta es la estructura estándar con la que la herramienta pretende trabajar, pero rápidamente aparecen varios problemas: 
    \newline 1.- El número de áreas varía a conveniencia de cada caso, normalmente la información listada anteriormente  aparece de manera mas o menos <<predecible>>, pero para su funcionamiento correcto hay que saber descartar las secciones inesperadas 
    \newline 2.- No todos los campos listados en cada área tienen por que contener información, es normal que hayan casillas en blanco
    \newline 3.- La sección donde se listan los documentos puede no contener la información que necesita el programa y en caso de contenerla es importante discriminar la versión actualizada de la antigua.
    \newline 4.- Ya habiendo identificado el hipervínculo correcto donde se contienen todos los documentos que busca el programa nos encontramos con una de las mayores problemáticas de la plataforma, la estandarización del Anexo I. Una licitación SIEMPRE debe contener este documento, ya que es el que (de manera relativamente estandarizada) contiene gran parte de la información relevante de cada proyecto, el problema viene a la hora de identificarlo. Suele ubicarse de dos formas: dentro del pliego administrativo (el cual es un documento con mucha información ajena al programa la cual hay que filtrar) o bajo el nombre de Anexo I; la forma mas eficiente de obtener este documento es buscando directamente el archivo con el nombre de Anexo I, sin embargo esta es otra área que no está estandarizada, ya que existen muchos casos en los que el archivo bajo ese nombre no es el esperado, y la única forma de identificarlo con seguridad es accediendo directamente al archivo, complicando aun más el proceso. 
    \newline 5.- La última problemática relacionada con la obtención del Anexo I es el formato del archivo en si, el programa toma en cuenta archivos .docx, odt y pdf, queda como trabajo futuro poder procesar archivos .zip o casos aun no contemplados.
    
    \item \textbf{Estructura y vocabulario:} 
    Cada documento de Anexo I cuenta con una estructura y un vocabulario completamente distintos dependiendo de la entidad emisora, si bien en muchas ocasiones contienen información similar, la forma en la que se presenta esta información tiene tantas variables que representa un reto enorme clasificar e identificar correctamente cada campo.
    
    \item \textbf{Factores externos:} 
    Toda la extracción y recopilación de información depende de un agente externo como lo es la página de contrataciondelestado.es, en caso de que esta falle o tarde demasiado en procesar las peticiones es razonable asumir un fallo de parte del programa.
    
\end{enumerate}

\subsubsection{Soluciones propuestas}
\begin{enumerate}
    \begin{table}[h!]
    \centering
    \begin{tabular}{|p{6cm}|p{6cm}|}
    \hline
    \textbf{MongoDB (NoSQL)} & \textbf{SQL} \\
    \hline
    \textbf{Pros:}
    \begin{itemize}
        \item Esquema flexible, ideal para datos no estructurados o cambiantes.
        \item Escalabilidad horizontal sencilla.
        \item Buen rendimiento para grandes volúmenes de datos y escritura rápida.
        \item Fácil integración con aplicaciones modernas y JSON.
    \end{itemize}
    &
    \textbf{Pros:}
    \begin{itemize}
        \item Integridad y consistencia de los datos gracias a transacciones ACID.
        \item Ideal para datos estructurados y relaciones complejas.
        \item Potente lenguaje de consultas (SQL) para búsquedas y análisis complejos.
        \item Amplio soporte, madurez y herramientas de administración.
    \end{itemize}
    \\
    \hline
    \textbf{Contras:}
    \begin{itemize}
        \item No garantiza consistencia fuerte por defecto (eventual consistency).
        \item Consultas complejas y relaciones entre datos menos eficientes.
        \item Menor soporte para transacciones complejas.
        \item Menos maduro para integridad referencial.
    \end{itemize}
    &
    \textbf{Contras:}
    \begin{itemize}
        \item Esquema rígido, menos flexible ante cambios frecuentes en la estructura.
        \item Escalabilidad horizontal más compleja.
        \item Puede ser menos eficiente para grandes volúmenes de datos no estructurados.
        \item Requiere definir relaciones y tipos de datos desde el inicio.
    \end{itemize}
    \\
    \hline
    \end{tabular}
    \caption{Comparativa de pros y contras entre MongoDB (NoSQL) y SQL}
    \end{table}
    
    \textbf{Justificación de la elección:}
    
    La base de datos seleccionada fue SQL porque la información a almacenar es estructurada, requiere integridad y relaciones claras entre entidades, y es fundamental poder realizar consultas complejas y precisas. SQL proporciona un entorno robusto y seguro para garantizar la consistencia de los datos, lo que es esencial para la trazabilidad y fiabilidad del sistema. Aunque MongoDB ofrece ventajas en flexibilidad y escalabilidad, las necesidades del proyecto se alinean mejor con las fortalezas de una base de datos relacional.

    \item  \textbf{Tecnología de extracción:} \newline 
    El código utiliza la librería de Python \texttt{playwright}, una de las más populares y actualizadas para la automatización y scraping de páginas web. Se compara aquí con \texttt{Beautiful Soup}, una alternativa clásica para el procesamiento de HTML.
    
    \begin{tabular}{p{7cm} p{7cm}}
    \textbf{Playwright} & \textbf{Beautiful Soup} \\
    \textbf{Pros:}
    \begin{itemize}
        \item Consistencia: Permite interactuar con páginas web dinámicas, ejecutando JavaScript y accediendo a elementos generados en tiempo real.
        \item Facilidad de uso: Proporciona una API moderna y completa para navegar, hacer clic, rellenar formularios y esperar eventos.
        \item Automatización avanzada: Puede simular acciones de usuario y manejar múltiples pestañas o contextos.
    \end{itemize}
    &
    \textbf{Pros:}
    \begin{itemize}
        \item Simplicidad: Muy fácil de aprender y usar para extraer información de HTML estático.
        \item Ligereza: Requiere pocos recursos y no necesita un navegador real para funcionar.
        \item Rápida para tareas sencillas: Ideal para scraping de páginas simples y bien estructuradas.
    \end{itemize}
    \\
    \textbf{Contras:}
    \begin{itemize}
        \item Requiere más recursos: Necesita instalar navegadores y puede consumir más memoria y CPU.
        \item Dependencia de la estructura: Cambios en el HTML pueden romper los selectores.
        \item Más compleja de instalar y configurar que Beautiful Soup.
    \end{itemize}
    &
    \textbf{Contras:}
    \begin{itemize}
        \item No ejecuta JavaScript: Solo puede analizar el HTML recibido, por lo que no accede a contenido dinámico generado en el navegador.
        \item Limitada para páginas modernas: No puede interactuar con formularios, botones o eventos de usuario.
        \item Menos robusta ante cambios en la estructura de la web.
    \end{itemize}
    \\
    \end{tabular}
    
    \textbf{Justificación de la elección:} \newline
    Se ha optado por \texttt{playwright} porque el proyecto requiere extraer información de páginas web con contenido dinámico y estructuras complejas, donde Beautiful Soup no sería suficiente. Playwright permite simular la interacción real de un usuario y acceder a todos los elementos de la página, garantizando una mayor robustez y flexibilidad en el scraping.


    
    \item  \textbf{Estructura y vocabulario:}
    Búsqueda semántica, RAG, modelo entrenado
    \textbf{Pros:} 
    \begin{itemize}
    \item Replicabilidad: 
    \item Escalabilidad: 
    \item Independencia de la estructura: 
    \end{itemize}
    \textbf{Contras:} 
    \begin{itemize}
    \item Posibilidad de fallo:
    \item Coste computacional: 
    \end{itemize}
    
    \item  \textbf{Factores externos:} \newline{}
    Al no existir garantías de servicio de parte de la web, se han incluido diversas técnicas de reconocimiento y tratamiento de errores dentro del programa.
    \newline{}
    \textbf{Pros:} 
    \begin{itemize}
    \item Recuperación: En ya mayoría de casos de error identificados existen protocolos para identificar, tratar y solucionar los errores manifestados.    \item Loggings: Todas las funciones que dependen de un servidor externo están contenidas en bloques de código que capturan correctamente el error y lo guardan en un archivo con fecha, hora, ubicación y causa del mismo, facilitando el trabajo de la persona asignada al mantenimiento.
    \end{itemize}
    \textbf{Contras:} 
    \begin{itemize}
    \item Fallos totales: En caso de que la web deje de entregar servicio es imposible extraer nuevos expedientes
    \end{itemize}
    

\end{enumerate}



\subsection{Solución propuesta}
\subsection{Plan de Trabajo}
\subsection{Diseño de la solución}
\subsubsection{Arquitectura del Sistema}
\subsubsection{Diseño Detallado}
\subsection{Tecnología Utilizada}

\subsubsection{RAG}

\begin{figure}[h!]
\centering
\includegraphics[width=15cm]{RAG_flow}
\caption{Imagen ilustrativa sobre el funcionamiento del RAG. Fuente: \url{https://www.gobierto.es/}}
\label{fig:RAG_flow}
\end{figure}

\subsubsection{Base Vectorial} 
\subsubsection{LLM} \label{LLM}
Como ya se mencionó anteriormente, el LLM elegido para este proyecto fue \textbf{Llama 3.3 70B Instruct}, esto debido principalmente a su buen desempeño como LLM de código abierto y la disponibilidad de recursos otorgados por la universidad. 
Según \cite{llm-stats}, este modelo se sitúa globalmente en el puesto 21 como uno de los mejores actualmente, pero si decidimos filtrar este número por modelos de código abierto, lo situamos en el puesto número 6, por detrás de modelos como \textbf{DeepSeek-R1}, \textbf{QwQ-32B} o \textbf{Phi-4}, para analizar la capacidad de este modelo, podemos hacer uso de la siguiente tabla comparandolo con el \textit{'estado del arte'} de los LLM de código abierto. 

\begin{table}[H]
\centering
\resizebox{\linewidth}{!}{%
\begin{tabular}{|l|c|c|}
\hline
\textbf{Característica} & \textbf{DeepSeek-R1}  & \textbf{Llama 3.3 70B Instruct} \\
\hline
Parametros & 671 & 70 \\
Contexto & 131.072 & 128.000 \\
1M input/\$ & \$0.55 & \$0.20 \\
1M output/\$ & \$2.19 & \$0.20 \\
GPQA & 71.5\% & 50.5\% \\
MMLU & 90.8\% & 86.0\% \\

\hline
\end{tabular}
}
\caption{Comparativa}
\label{tab:comparativa_llm}
\end{table}

\subsubsection{Scrapper}
\subsubsection{SQLServer}
\subsubsection{Python}
\subsubsection{Django}

\subsection{Desarrollo de la solución propuesta}
\newpage

\section{Implementación}
Lorem ipsum dolor sit amet consectetur adipiscing elit, duis nostra sagittis nunc class mauris fermentum, semper lobortis eu dui per ridiculus. Sodales augue ad neque lobortis taciti facilisi, nec cum vehicula scelerisque senectus ante, inceptos massa maecenas vel natoque. Faucibus sem mattis sociosqu tempor proin sapien egestas tempus, purus condimentum ligula tellus libero penatibus mauris tortor, sagittis cum aenean nunc rutrum odio habitasse.

Eros sociis dictumst auctor habitasse libero molestie nascetur laoreet sodales, a vitae cubilia sollicitudin hendrerit elementum neque ullamcorper, mollis ultrices felis enim conubia lacus scelerisque mi. Semper orci nisl aliquam imperdiet viverra ac, molestie litora penatibus aliquet himenaeos feugiat conubia, habitasse eu leo volutpat curae. Quam parturient purus accumsan eu dui curae torquent porta ligula, nibh ornare augue aenean mus sem iaculis arcu, et sapien eros volutpat enim feugiat ac metus.
\newpage

\section{Pruebas}
Lorem ipsum dolor sit amet consectetur adipiscing elit, duis nostra sagittis nunc class mauris fermentum, semper lobortis eu dui per ridiculus. Sodales augue ad neque lobortis taciti facilisi, nec cum vehicula scelerisque senectus ante, inceptos massa maecenas vel natoque. Faucibus sem mattis sociosqu tempor proin sapien egestas tempus, purus condimentum ligula tellus libero penatibus mauris tortor, sagittis cum aenean nunc rutrum odio habitasse.

Eros sociis dictumst auctor habitasse libero molestie nascetur laoreet sodales, a vitae cubilia sollicitudin hendrerit elementum neque ullamcorper, mollis ultrices felis enim conubia lacus scelerisque mi. Semper orci nisl aliquam imperdiet viverra ac, molestie litora penatibus aliquet himenaeos feugiat conubia, habitasse eu leo volutpat curae. Quam parturient purus accumsan eu dui curae torquent porta ligula, nibh ornare augue aenean mus sem iaculis arcu, et sapien eros volutpat enim feugiat ac metus.
\newpage

\section{Conclusiones}
Lorem ipsum dolor sit amet consectetur adipiscing elit, duis nostra sagittis nunc class mauris fermentum, semper lobortis eu dui per ridiculus. Sodales augue ad neque lobortis taciti facilisi, nec cum vehicula scelerisque senectus ante, inceptos massa maecenas vel natoque. Faucibus sem mattis sociosqu tempor proin sapien egestas tempus, purus condimentum ligula tellus libero penatibus mauris tortor, sagittis cum aenean nunc rutrum odio habitasse.

Eros sociis dictumst auctor habitasse libero molestie nascetur laoreet sodales, a vitae cubilia sollicitudin hendrerit elementum neque ullamcorper, mollis ultrices felis enim conubia lacus scelerisque mi. Semper orci nisl aliquam imperdiet viverra ac, molestie litora penatibus aliquet himenaeos feugiat conubia, habitasse eu leo volutpat curae. Quam parturient purus accumsan eu dui curae torquent porta ligula, nibh ornare augue aenean mus sem iaculis arcu, et sapien eros volutpat enim feugiat ac metus.
\newpage

\section{Relación del trabajo desarrollado con los estudios cursados}
\begin{itemize}
    \item \textbf{Interfaces persona computador:}  
    Esta asignatura fue nuestro primer acercamiento al mundo del frontend, nos otorgó nociones básicas de lo que es una buena interfaz de usuario y lo que no, gracias a ella pudimos obtener bases sólidas sobre cómo diseñar correctamente un programa dependiendo del usuario objetivo.

    \item \textbf{Bases de datos y sistemas de información:}  
    Me otorgó una base bastante sólida sobre el diseño, mantenimiento, extracción e inserción de información en una base de datos mediante el uso de SQL. Sin esta asignatura no hubiera sido posible manejar el volumen de datos que el programa requiere de una forma escalable y eficiente.

    \item \textbf{Ingeniería del software:}  
    Si bien al momento de tomar esta asignatura, la mayoría de los alumnos ya teníamos una idea de cómo hacer y diseñar un programa funcional, fue esta materia la que nos dio las herramientas correctas para saber diseñar software de manera sostenible y estandarizada, enseñándonos a cómo trabajar en equipo, las capas que debería tener un proyecto, formas de mantener cada elemento independiente y encapsulado, técnicas para analizar y solucionar problemas, y un largo etcétera.

    \item \textbf{Tecnología de sistemas de información en la red:}  
    Herramientas como Django basan su lógica en buena parte del temario de esta materia, sin ella no hubiera sido capaz de comprender correctamente la funcionalidad de los servidores - clientes - proxys, cómo se manejan las solicitudes de cada usuario, cómo se maneja más de un usuario para un solo servidor, etcétera.

    \item \textbf{Sistemas de almacenamiento y recuperación de información:}  
    Concretamente, \textbf{SAR} fue una de las asignaturas de las que más siento haber sacado provecho para este proyecto concreto, ya que nos enseñó a diseñar un scrapper (parte indispensable de este proyecto), un programa de generación de texto (muy útil para comprender el funcionamiento de un LLM), funcionamiento de los embeddings y algoritmos de procesamiento de texto variados.

    \item \textbf{Percepción / Aprendizaje automático:}  
    Combino ambas materias debido a que una es la sucesora de la otra y han sido igual de importantes para este proyecto. Ellas me dieron una base muy sólida para comprender cómo funciona un LLM desde dentro, acercarme al mundo de las redes neuronales y sistemas de aprendizaje automático hubiera sido infinidad de veces más complicado sin los conocimientos que logré adquirir gracias a ambas asignaturas.
\end{itemize}



\newpage

\section{Trabajos futuros}
Lorem ipsum dolor sit amet consectetur adipiscing elit, duis nostra sagittis nunc class mauris fermentum, semper lobortis eu dui per ridiculus. Sodales augue ad neque lobortis taciti facilisi, nec cum vehicula scelerisque senectus ante, inceptos massa maecenas vel natoque. Faucibus sem mattis sociosqu tempor proin sapien egestas tempus, purus condimentum ligula tellus libero penatibus mauris tortor, sagittis cum aenean nunc rutrum odio habitasse.

Eros sociis dictumst auctor habitasse libero molestie nascetur laoreet sodales, a vitae cubilia sollicitudin hendrerit elementum neque ullamcorper, mollis ultrices felis enim conubia lacus scelerisque mi. Semper orci nisl aliquam imperdiet viverra ac, molestie litora penatibus aliquet himenaeos feugiat conubia, habitasse eu leo volutpat curae. Quam parturient purus accumsan eu dui curae torquent porta ligula, nibh ornare augue aenean mus sem iaculis arcu, et sapien eros volutpat enim feugiat ac metus.
\newpage

\section{Referencias}
\cite{3blue1brown}
\printbibliography

\newpage

\section{Anexos}
Lorem ipsum dolor sit amet consectetur adipiscing elit, duis nostra sagittis nunc class mauris fermentum, semper lobortis eu dui per ridiculus. Sodales augue ad neque lobortis taciti facilisi, nec cum vehicula scelerisque senectus ante, inceptos massa maecenas vel natoque. Faucibus sem mattis sociosqu tempor proin sapien egestas tempus, purus condimentum ligula tellus libero penatibus mauris tortor, sagittis cum aenean nunc rutrum odio habitasse.

Eros sociis dictumst auctor habitasse libero molestie nascetur laoreet sodales, a vitae cubilia sollicitudin hendrerit elementum neque ullamcorper, mollis ultrices felis enim conubia lacus scelerisque mi. Semper orci nisl aliquam imperdiet viverra ac, molestie litora penatibus aliquet himenaeos feugiat conubia, habitasse eu leo volutpat curae. Quam parturient purus accumsan eu dui curae torquent porta ligula, nibh ornare augue aenean mus sem iaculis arcu, et sapien eros volutpat enim feugiat ac metus.

\end{document}